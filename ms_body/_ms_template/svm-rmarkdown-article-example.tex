\documentclass[12pt,]{article}
\usepackage[left=1in,top=1in,right=1in,bottom=1in]{geometry}
\newcommand*{\authorfont}{\fontfamily{phv}\selectfont}
\usepackage[]{mathpazo}


  \usepackage[T1]{fontenc}
  \usepackage[utf8]{inputenc}




\usepackage{abstract}
\renewcommand{\abstractname}{}    % clear the title
\renewcommand{\absnamepos}{empty} % originally center

\renewenvironment{abstract}
 {{%
    \setlength{\leftmargin}{0mm}
    \setlength{\rightmargin}{\leftmargin}%
  }%
  \relax}
 {\endlist}

\makeatletter
\def\@maketitle{%
  \newpage
%  \null
%  \vskip 2em%
%  \begin{center}%
  \let \footnote \thanks
    {\fontsize{18}{20}\selectfont\raggedright  \setlength{\parindent}{0pt} \@title \par}%
}
%\fi
\makeatother




\setcounter{secnumdepth}{0}




 



\author{}


\date{}

\usepackage{titlesec}

\titleformat*{\section}{\normalsize\bfseries}
\titleformat*{\subsection}{\normalsize\itshape}
\titleformat*{\subsubsection}{\normalsize\itshape}
\titleformat*{\paragraph}{\normalsize\itshape}
\titleformat*{\subparagraph}{\normalsize\itshape}





\newtheorem{hypothesis}{Hypothesis}
\usepackage{setspace}


% set default figure placement to htbp
\makeatletter
\def\fps@figure{htbp}
\makeatother

\usepackage{hyperref}
\setlength\parindent{24pt}
\usepackage{multirow}
\usepackage{booktabs}
\usepackage{longtable}
\usepackage{array}
\usepackage{multirow}
\usepackage{wrapfig}
\usepackage{float}
\usepackage{colortbl}
\usepackage{pdflscape}
\usepackage{tabu}
\usepackage{threeparttable}
\usepackage{threeparttablex}
\usepackage[normalem]{ulem}
\usepackage{makecell}
\usepackage{xcolor}

% move the hyperref stuff down here, after header-includes, to allow for - \usepackage{hyperref}

\makeatletter
\@ifpackageloaded{hyperref}{}{%
\ifxetex
  \PassOptionsToPackage{hyphens}{url}\usepackage[setpagesize=false, % page size defined by xetex
              unicode=false, % unicode breaks when used with xetex
              xetex]{hyperref}
\else
  \PassOptionsToPackage{hyphens}{url}\usepackage[draft,unicode=true]{hyperref}
\fi
}

\@ifpackageloaded{color}{
    \PassOptionsToPackage{usenames,dvipsnames}{color}
}{%
    \usepackage[usenames,dvipsnames]{color}
}
\makeatother
\hypersetup{breaklinks=true,
            bookmarks=true,
            pdfauthor={},
             pdfkeywords = {},  
            pdftitle={},
            colorlinks=true,
            citecolor=blue,
            urlcolor=blue,
            linkcolor=magenta,
            pdfborder={0 0 0}}
\urlstyle{same}  % don't use monospace font for urls

% Add an option for endnotes. -----


% add tightlist ----------
\providecommand{\tightlist}{%
\setlength{\itemsep}{0pt}\setlength{\parskip}{0pt}}

% add some other packages ----------

% \usepackage{multicol}
% This should regulate where figures float
% See: https://tex.stackexchange.com/questions/2275/keeping-tables-figures-close-to-where-they-are-mentioned
\usepackage[section]{placeins}


\begin{document}
	
% \pagenumbering{arabic}% resets `page` counter to 1 
%    





\vskip -8.5pt


 % removetitleabstract

\noindent \doublespacing 

\pagebreak

\setlength{\parindent}{0in}
\setlength{\leftskip}{0in}
\setlength{\parskip}{8pt}
\vspace*{-0.2in}

\noindent

\textbf{Cryptic diversity and host-specificity of \emph{Tetramesa}
(Hymenoptera: Eurytomidae) associated with two African grasses:
implications for biological control}

Sutton, G. F.\textsuperscript{1}\(\dag\), van Steenderen,
C.\textsuperscript{1}, Canavan, K\textsuperscript{1}, Day, M.
D.\textsuperscript{2}, Paterson, I. D.\textsuperscript{1}

~

\begingroup
\fontsize{10}{12}\selectfont

\textsuperscript{1} Center for Biological Control, Department of Zoology
and Entomology, Rhodes University, Makhanda, 6140, South Africa

\textsuperscript{2} Department of Agriculture and Fisheries, GPO Box
267, Brisbane, Australia \endgroup

~

\textbf{Corresponding author} \(\dag\)
\href{mailto:g.sutton@ru.ac.za}{\nolinkurl{g.sutton@ru.ac.za}}

\pagebreak

\setlength\parindent{24pt}

\hypertarget{abstract}{%
\section{Abstract}\label{abstract}}

\emph{Sporobolus pyramidalis} Beauv. and \emph{Sporobolus natalensis}
(Steud.) Th. Dur. \& Schinz. (Poaceae) (giant rat tail grass), are two
perennial African grasses that have become invasive in Australia. Field
host-range surveys on 47 non-target grasses in the weeds native
distribution (South Africa), between 2017 and 2019, found an apparently
specialised and highly damaging stem-boring wasp, \emph{Tetramesa} sp.
1. Subsequent field surveys in late 2019 found a morphologically
indistinguishable \emph{Tetramesa} sp. on a closely-related non-target
grass, \emph{Eragrostis curvula} (Schrad.) Nees (African lovegrass). If
the wasps reared from the different host-plants (\emph{S. pyramidalis}
or \emph{E. curvula}) represent the same genetic entity, this would
preclude its use as a biological control agent. In this study, we showed
that two morphologically cryptic \emph{Tetramesa} sp. populations
collected from \emph{E. curvula} and \emph{S. pyramidalis},
respectively, represent two distinct genetic clades, likely representing
distinct species. Cross-inoculation trials indicated that each wasp
could survive on the host plant off which it was collected. Our findings
indicate that practitioners need to be aware of cryptic species when
conducting field host-range surveys. Morphologically indistinguishable
candidate agents might in fact be crytic species with their own
restricted host ranges, and if we do not take this into account, then we
may exclude potentially effective and host-specific agents.

~

\setlength{\parindent}{0in}
\setlength{\leftskip}{0in}
\setlength{\parskip}{8pt}
\vspace*{-0.2in}

\noindent

\textbf{Keywords}: cryptic species; DNA barcoding; \emph{Eragrostis
curvula}; giant rat's tail grass'; invasive grass; \emph{Sporobolus}

\pagebreak
\setlength\parindent{24pt}

\hypertarget{introduction}{%
\section{1. Introduction}\label{introduction}}

\emph{Sporobolus pyramidalis} Beauv.

Recent estimates indicate that there are approximately 7 million insect
species on earth, albeit only about 20\% have been described to date
(Stork, 2018). There are strong geographic and taxonomic biases in
sampling coverage and intensity, with Afrotropical Hymenoptera being
particularly under-studied and poorly known (van Noort et al., 2015).
There are approximately 205 described \emph{Tetramesa} species known,
worldwide (Al-Barrak et al., 2004), although only four species have been
described in the Afroptropical region, none of which are known from
South Africa (van Noort, 2019). Our study reports on two potentially
novel \emph{Tetramesa} species from South Africa. While these wasps have
not been formally described yet, these specimens have been sent to an
expert taxonomist for an official scientific description, but our
\emph{COI} barcoding and rearing trials provide strong support for the
species status of both \emph{Tetramesa} identified during the current
study. An additional five putative \emph{Tetramesa} species have been
identified during our field surveys in South Africa (G.F. Sutton,
unpublished data), and are currently undergoing further study. Surveys
for potential biological control agents of weeds provide an exciting
opportunity for documenting and discovering previously unknown
biodiversity, which is particularly relevant for surveys performed in
Africa and other relatively understudied geographic regions.

\hypertarget{acknowledgements}{%
\section{Acknowledgements}\label{acknowledgements}}

\newpage

\hypertarget{references}{%
\section{References}\label{references}}

\setlength{\parindent}{-0.2in}
\setlength{\leftskip}{0.2in}
\setlength{\parskip}{8pt}
\vspace*{-0.2in}

\noindent

\newlength{\cslhangindent}
\setlength{\cslhangindent}{1.5em}
\newenvironment{CSLReferences}%
{\setlength{\parindent}{0pt}%
\everypar{\setlength{\hangindent}{\cslhangindent}}\ignorespaces}%
{\par}

\hypertarget{refs}{}
\begin{CSLReferences}{1}{0}
\leavevmode\hypertarget{ref-Al-Barrak2004}{}%
Al-Barrak, M., Loxdale, H.D., Brookes, C.P., Dawah, H.A., Biron, D.G.,
Alsagair, O., 2004. Molecular evidence using enzyme and {RAPD} markers
for sympatric evolution in {British} species of {\emph{Tetramesa}}
({Hymenoptera}: {Eurytomidae}). Biological Journal of the Linnean
Society 83, 509--525.
\url{https://doi.org/10.1111/j.1095-8312.2004.00408.x}

\leavevmode\hypertarget{ref-Stork2018}{}%
Stork, N.E., 2018. How many species of insects and other terrestrial
arthropods are there on earth? Annual Review of Entomology 63, 31--45.

\leavevmode\hypertarget{ref-VanNoort2015}{}%
van Noort, S., Buffington, M.L., Forshage, M., 2015. Afrotropical
{Cynipoidea} ({Hymenoptera}). ZooKeys 1--176.
\url{https://doi.org/10.3897/zookeys.493.6353}

\end{CSLReferences}

\newpage

\hypertarget{figure-and-table-legend}{%
\section{Figure and table legend}\label{figure-and-table-legend}}

\setlength{\parindent}{-0.2in}
\setlength{\leftskip}{0.2in}
\setlength{\parskip}{8pt}
\vspace*{-0.2in}

\noindent

\textbf{Table 1} Results from no-choice cross-inoculation trials.

\textbf{Table 2} Results from no-choice greenhouse-based host-range
trials for \emph{Tetramesa} sp. reared from \emph{Sporobolus
pyramidalis} and \emph{S. natalensis}.

\textbf{Figure 1} Genetics of \emph{Tetramesa} spp.

\newpage





\newpage
\singlespacing 
\end{document}
